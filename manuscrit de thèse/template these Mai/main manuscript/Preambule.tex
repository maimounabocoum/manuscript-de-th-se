
%
%%%%%%%%%%%%%%%%%%%%%%%%%%%%%%%%%%%%%%%%%
%%           Liste des packages         %
%%%%%%%%%%%%%%%%%%%%%%%%%%%%%%%%%%%%%%%%%

\usepackage[english, francais]{babel}  
\usepackage[utf8]{inputenc}
\usepackage[T1]{fontenc}
\usepackage{cite}

%\usepackage[babel=true]{csquotes}

\usepackage{color}

\usepackage[normalem]{ulem}
\usepackage{soul}
%\usepackage[pdftex,colorlinks=true,linkcolor=blue,citecolor=green,pagebackref=true]{hyperref}
\usepackage[colorlinks=true,linkcolor=blue,citecolor=lblue,pagebackref]{hyperref}

\usepackage[french]{minitoc}
\usepackage{textcomp}
\usepackage{lmodern}
\usepackage{xspace}
\usepackage[final]{pdfpages}
\usepackage{appendix}
\usepackage{geometry}  
\usepackage[width=0.9\textwidth]{caption}
\usepackage{array}
\usepackage{url}
\usepackage{fancyhdr}
% permet de faire une table des matieres par chapitre
\usepackage{lipsum}
\usepackage[french]{minitoc}


%%%%%%%%%%%%%%%%%%%%%%%%%%%%%%%%%%%%%%%%%
%%           new commands         %
%%%%%%%%%%%%%%%%%%%%%%%%%%%%%%%%%%%%%%%%%

%\newcommand{\com}[1]{{\color{red}\emph{#1}}}
\definecolor{lblue}{rgb}{0,0,0.2}

\newcommand{\alp}{\texorpdfstring{\ensuremath{\upalpha}\xspace}{alpha }}
\newcommand{\bet}{\texorpdfstring{\ensuremath{\upbeta}\xspace}{b\'{e}ta }}
\newcommand{\alpbet}{\texorpdfstring{\ensuremath{\upalpha-\upbeta}\xspace}{alpha-b\'{e}ta}}
\newcommand{\alpt}{\ensuremath{\alpha_2}\xspace}
\newcommand{\strt}{\gls{strt}\xspace}



\newcommand{\g}[1]{\textbf{#1}}  % à mettre en valeur
\newcommand{\T}[1]{\underline{#1}}
\newcommand{\TT}[1]{\underline{\underline{#1}}}
\newcommand{\TTT}[1]{\underline{\underline{\underline{#1}}}}



%%%%%%%%%%%%%%%%%%%%%%%%%%%%%%%%%%%%%%%%%%%%%%%%%%%%%%%%%%%%%%%%%%%%%%

%%\usepackage[square,sort&compress,sectionbib]{natbib}		% Doit être chargé avant babel
%\usepackage[square,sort,comma,numbers]{natbib}
%\usepackage{chapterbib}
%	\renewcommand{\bibsection}{\section{Références}}		% Met les références biblio dans un \section (au lieu de \section*)
%		

%\usepackage{lmodern}
%\usepackage{ae,aecompl}										% Utilisation des fontes vectorielles modernes
%\usepackage[upright]{fourier}
%
%
%
%%%%%%%%%%%%%%%%%%%%%%%%%%%%%%%%%%%%%%%%%%%%%%%%%%%%%%%%%%%%%%%%%%%%%%
%
%%%%%%%%%%%%%%%%%%%%%%%%%%%%%%%%%%%%%%%%%
%%           Apparence         %
%%%%%%%%%%%%%%%%%%%%%%%%%%%%%%%%%%%%%%%%%          
\pagestyle{fancy}
\fancyhf{}
\renewcommand{\chaptermark}[1]{\markboth{\bsc{\chaptername~\thechapter{} :} #1}{}}
\renewcommand{\sectionmark}[1]{\markright{\thesection{} \ #1}}
\renewcommand{\headrulewidth}{0pt}
\lhead[]{\textsl{\rightmark}}
\rhead[\textsl{\leftmark}]{}
\cfoot[\thepage]{\thepage}

\NoAutoSpaceBeforeFDP





%%%%%%%%%%%%%%%%%%%%%%%%%%%%%%%%%%%%%%%%%%%%%%%%%%%%%%%%%%%%%%%%%%%%%%
%
%%% Maths                         
\usepackage{amsmath}			% Permet de taper des formules mathématiques
\usepackage{amssymb}			% Permet d'utiliser des symboles mathématiques
\usepackage{amsfonts}			% Permet d'utiliser des polices mathématiques
\usepackage{nicefrac}
\usepackage{upgreek}			% For roman (i.e. upright) lowercase Greek characters
%
%%%%%%%%%%%%%%%%%%%%%%%%%%%%%%%%%%%%%%%%%%%%%%%%%%%%%%%%%%%%%%%%%%%%%%
%
%
%%% Tableaux
\usepackage{multirow}
\usepackage{booktabs}
\usepackage{colortbl}
\usepackage{tabularx}
\usepackage{multirow}
\usepackage{threeparttable}
\usepackage{etoolbox}
%	\appto\TPTnoteSettings{\footnotesize}
%\addto\captionsfrench{\def\tablename{{\textsc{Tableau}}}}	% Renome 'table' en 'tableau'
%
%            
%            
%
%%%%%%%%%%%%%%%%%%%%%%%%%%%%%%%%%%%%%%%%%%%%%%%%%%%%%%%%%%%%%%%%%%%%%%
%%% Graphiques         
           
\usepackage{subfig}
\usepackage{graphicx}
\usepackage{epstopdf}

%\usepackage{subcaption}
%\usepackage{pdfpages}
%\usepackage{rotating}
%\usepackage{pgfplots}
%	\usepgfplotslibrary{groupplots}
%\usepackage{tikz}
%	\usetikzlibrary{backgrounds,automata}
%	\pgfplotsset{width=7cm,compat=1.3}
%	\tikzset{every picture/.style={execute at begin picture={
%   		\shorthandoff{:;!?};}
%	}}
%	\pgfplotsset{every linear axis/.append style={
%		/pgf/number format/.cd,
%		use comma,
%		1000 sep={\,},
%	}}
%\usepackage{eso-pic}
%\usepackage{import}
%\usepackage{cclicenses}
%
%%%%%%%%%%%%%%%%%%%%%%%%%%%%%%%%%%%%%%%%%%%%%%%%%%%%%%%%%%%%%%%%%%%%%%
%% Biblio                        
%%\makeatletter
%%\patchcmd{\BR@backref}{\newblock}{\newblock(page~}{}{}	% Pour les back-references, affiche 'page' au lieu de 'p.'
%%\patchcmd{\BR@backref}{\par}{)\par}{}{}
%%\makeatother
%	
%	
%%%%%%%%%%%%%%%%%%%%%%%%%%%%%%%%%%%%%%%%%%%%%%%%%%%%%%%%%%%%%%%%%%%%%%
%%% Navigation dans le document   
%\usepackage[pdftex,pdfborder={0 0 0},
%			colorlinks=true,
%			linkcolor=blue,
%			citecolor=red,
%			pagebackref=true,
%			]{hyperref} %Créera automatiquement les liens internes au PDF
%
%
%%%%%%%%%%%%%%%%%%%%%%%%%%%%%%%%%%%%%%%%%%%%%%%%%%%%%%%%%%%%%%%%%%%%%%
%%% Mise en forme du texte        
%\usepackage{xspace}
%\usepackage[load-configurations = abbreviations]{siunitx}
%	\DeclareSIUnit{\MPa}{\mega\pascal}
%	\DeclareSIUnit{\micron}{\micro\meter}
%	\DeclareSIUnit{\tr}{tr}
%	\DeclareSIPostPower\totheM{m}
%	\sisetup{
%	locale = FR,
%	  inter-unit-separator=$\cdot$,
%	  range-phrase=~\`{a}~,     	% Utilise le tiret court pour dire "de... à"
%	  range-units=single,  			% Cache l'unité sur la première borne
%	  }
%\usepackage{chemist}
%\usepackage[version=3]{mhchem}
%\usepackage{textcomp}
%\usepackage{numprint}
%\usepackage{array}
%\usepackage[acronym,xindy,toc,numberedsection]{glossaries}
%	\newglossary[nlg]{notation}{not}{ntn}{Notation} 	% Création d'un type de glossaire 'notation'
%	\makeglossaries
%	\loadglsentries{Glossaire}							% Utilisation d'un fichier externe pour la définition des entrées (Glossaire.tex)
%\usepackage{hyphenat}
%
%
%%% Rajouter en plus du template pour moi :
\usepackage{lipsum}
\usepackage{scrhack}
\usepackage{amssymb}
\usepackage{mathrsfs,amsthm}
\usepackage{calrsfs}
\usepackage{mathtools}
\usepackage[overload]{empheq}
\usepackage{cases} 
\usepackage{multicol}
\usepackage{makeidx}
\usepackage[nottoc]{tocbibind} % ajoute (entre autre) la bibliographie dans la table des matieres 
\usepackage{float}
\usepackage{braket}
%\usepackage[font=small,labelfont=bf,justification=justified]{caption}
%
%%%%%%%%%%%%%%%%%%%%%%%%%%%%%%%%%%%%%%%%%%%%%%%%%%%%%%%%%%%%%%%%%%%%%%
%%% Compilation
%
%\usepackage{silence}
%
%
%%% Virer les erreur dues à minitoc
%\WarningFilter{minitoc(hints)}{W0023}
%\WarningFilter{minitoc(hints)}{W0024}
%\WarningFilter{minitoc(hints)}{W0028}
%\WarningFilter{minitoc(hints)}{W0030}
%
% 
%
%
%%%%%%%%%%%%%%%%%%%%%%%%%%%%%%%%%%%%%%%%%
%%           Page de garde              %
%%%%%%%%%%%%%%%%%%%%%%%%%%%%%%%%%%%%%%%%%
\makeatletter
\def\@ecole{école}
\newcommand{\ecole}[1]{
  \def\@ecole{#1}
}

\def\@specialite{Spécialité}
\newcommand{\specialite}[1]{
  \def\@specialite{#1}
}

\def\@directeur{directeur}
\newcommand{\directeur}[1]{
  \def\@directeur{#1}
}

\def\@encadrant{encadrant}
\newcommand{\encadrant}[1]{
  \def\@encadrant{#1}
}
\def\@jurya{}{}{}
\newcommand{\jurya}[3]{
  \def\@jurya{#1,	& #2	& #3\\}
}
\def\@juryb{}{}{}
\newcommand{\juryb}[3]{
  \def\@juryb{#1,	& #2	& #3\\}
}
\def\@juryc{}{}{}
\newcommand{\juryc}[3]{
  \def\@juryc{#1,	& #2	& #3\\}
}
\def\@juryd{}{}{}
\newcommand{\juryd}[3]{
  \def\@juryd{#1,	& #2	& #3\\}
}
\def\@jurye{}{}{}
\newcommand{\jurye}[3]{
  \def\@jurye{#1,	& #2	& #3\\}
}
\def\@juryf{}{}{}
\newcommand{\juryf}[3]{
  \def\@juryf{#1,	& #2	& #3\\}
}
\def\@juryg{}{}{}
\newcommand{\juryg}[3]{
  \def\@juryg{#1,	& #2	& #3\\}
}
\def\@juryh{}{}{}
\newcommand{\juryh}[3]{
  \def\@juryh{#1,	& #2	& #3\\}
}
\def\@juryi{}{}{}
\newcommand{\juryi}[3]{
  \def\@juryi{#1,	& #2	& #3\\}
}
\makeatother

\newcommand\BackgroundPic{%
	\put(0,0){%
		\parbox[b][\paperheight]{\paperwidth}{%
			\includegraphics[height=0.45\paperheight]{bordure.png}%
			\vfill
		}
	}
}
\newcommand\EtiquetteThese{%
	\put(0,0){%
		\parbox[t][\paperheight]{\paperwidth}{%
			\hfill
			\colorbox{blue}{		
				\begin{minipage}[b]{3em}
					\centering\Huge\textcolor{white}{T\\H\\E\\S\\E\\}
					\vspace{0.2cm}
				\end{minipage}
			}
		}
	}
}

\makeatletter
\newcommand{\pagedegarde}{
\newgeometry{top=2.5cm, bottom=2cm, left=2cm, right=1cm}
\AddToShipoutPicture*{\BackgroundPic}
\AddToShipoutPicture*{\EtiquetteThese}

  \begin{titlepage}
  \centering
      \includegraphics[width=0.4\textwidth]{ParisTech-Institute.pdf}
      \hfill
      \includegraphics[width=0.15\textwidth]{ENSTAlogo}\\
    \vspace{1cm}
      {\Large \'{E}cole doctorale Paris Saclay (EDX)}\\
    \vspace{1cm}
      {\huge 
      	{\bfseries Doctorat ParisTech}\\
    \vspace{0.5cm}
      	TH\`{E}SE}\\
    \vspace{1cm}
   		{\bfseries pour obtenir le grade de docteur délivré par}\\
    \vspace{1cm}
    	{\huge\bfseries \@ecole}\\
    \vspace{0.5cm}
    	{\Large{\bfseries Spécialité: Laser et Matière}}\\
    \vspace{2cm}
    	\textit{présentée et soutenue publiquement par}\\
    \vspace{0.5cm}
    	{\Large {\bfseries \@author}} \\
    \vspace{0.5cm}
    	le \@date \\
    \vfill
       {\LARGE \color[rgb]{0,0,1} \bfseries{\@title}} \\
    \vfill
        Directeur de thèse : {\bfseries \@directeur}\\
        Co-encadrant de thèse : {\bfseries \@encadrant}\\
    \vfill
	\begin{tabular}{>{\bfseries}llr}
		\large Jury\\
		\@jurya
		\@juryb
		\@juryc
		\@juryd
		\@jurye
		\@juryf
		\@juryg
		\@juryh
		\@juryi
	\end{tabular}
	\vfill
	
	\textbf{Thèse préparée au Laboratoire d'Optique Appliquée UMR 7639\\
	ENSTA ParisTech - Ecole Polytechnique - CNRS}

  \end{titlepage}




\restoregeometry  
  
  
}
\makeatother
